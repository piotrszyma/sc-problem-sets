\section{Zadanie 3}
\subsection{Opis problemu}
W tym zadaniu należało dla zadanych współczynników wielomianu interpolacyjnego w postaci Newtona $c_0 = f[x_0], c_1 = f[x_0, x_1], c_2 = f[x_0, x_1, x_2], \ldots, c_n = f[x_0, \ldots, x_n] $ (ilorazy różnicowe) oraz węzłów $x_0, x_1, \ldots, x_n$ napisać funkcję obliczającą w czasie $O(n^2)$ współczynniki $a_0, \ldots, a_n$ jego postaci naturalnej, tzn. $ a_nx^n + a_{n-1}x^{n-1} + \ldots + a_1x + a_0 $
\subsection{Analiza}
To zadanie bazowało na wynikach zadania 9 z listy 4 z ćwiczeń, gdzie pokazaliśmy, że mając wielomian w postaci Newtona, jesteśmy w stanie, przy pomocy wielomianów pomocniczych z metody Hornera, wyznaczyć współczynniki postaci naturalnej.
\subsection{Implementacja}
Algorytm rozpoczynamy od stworzenia tablicy $ \mathtt{A[n]}$, w której będziemy trzymać kolejne współczynniki. Zaczynamy od końca, do komórki $ \mathtt{A[n]}$ podstawiając iloraz różnicowy z $n$-tego węzła. Kolejne współczynniki obliczamy, biorąc wielomian dla wcześniejszego elementu i wyliczając wielomian pomocniczy dla tego, kolejnego współczynnika. W kolejnym kroku współczynniki z tego wielomianu pomocniczego zestawiamy z dotychczas wyliczonymi współczynnikami. Te kroki powtarzamy aż do momentu zejścia do współczynnika $a_0$. Po przejściu całego algorytmu otrzymujemy tabelę $\mathtt{A[n]}$ będącą wektorem współczynników wielomianu postaci normalnej. W każdej z $n$ iteracji algorytmu musimy dokonać zestawienia do $n$ współczynników. Złożoność algorytmu wynosi więc $O(n) * O(n)$, tj. $O(n^2)$. \\\\ 
Moja implementacja algorytmu w języku Julia znajduje się w module $ \texttt{MyModule} $ znajdującym się w pliku załączonym do tego sprawozdania.
