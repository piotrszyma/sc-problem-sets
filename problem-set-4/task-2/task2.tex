\section{Zadanie 2}
\subsection{Opis problemu}
Celem zadania było stworzenie funkcji obliczającej wartość wielomianu interpolacyjnego stopnia $ n $ w postaci Newtona $N_n(x)$ w punkcie $ x = t $ za pomocą uogólninonego algorytmu Hornera w czasie $ O(n) $. (implementacja zadania 8 z listy nr 4 z ćwiczeń)
\\ $ \texttt{function warNewton (x::Vector{Float64}, fx::Vector{Float64}, t::Float64)} $ \\
\subsection{Analiza}
\subsection{Implementacja}
Moja implementacja algorytmu w języku Julia znajduje się w module $ \texttt{MyModule} $ znajdującym się w pliku załączonym do tego sprawozdania.