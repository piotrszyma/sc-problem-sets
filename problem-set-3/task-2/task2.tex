\section{Zadanie 2}
\subsection{Opis problemu}
Zadanie polegało na zaimplementowaniu algorytmu rozwiązującego równanie $ f(x) = 0 $ $\textsc{metodą stycznych} $.
\subsection{Pseudokod}
\begin{algorithm}[H]
  \KwData{$f, x_{0}, M, \delta, \epsilon$}
  \KwResult{$x, f(x), i, err$}
  $v \leftarrow f(x_0);$\\
  \If{$ |v| < \epsilon $}{
    return (0, 0, 0, 1);
  }
  \Do{$ k < M $}{
    $ k\;+= 1 $ \\
    $ x_1 \leftarrow x_0 - \frac{v}{f'(x_0)}$ \\ 
    $ v \leftarrow f(x_1) $ \\
    \If{$ |x_1 - x_0| < \delta $ or $ |v| < \epsilon $}{
      return $ (x_1, v, i, 0) $;
    }
    $ x_0 \leftarrow x_1 $
  }
\end{algorithm}
\subsection{Implementacja}
Moja implementacja algorytmu w języku Julia znajduje się w module $ \texttt{MyModule} $ znajdującym się w pliku załączonym do tego sprawozdania.