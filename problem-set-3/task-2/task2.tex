\section{Zadanie 2}
\subsection{Opis problemu}
Zadanie polegało na zaimplementowaniu algorytmu rozwiązującego równanie $ f(x) = 0 $ $\textsc{metodą stycznych} $.
\subsection{Analiza}
\begin{algorithm}[H]
  \KwData{$f, x_{0}, M, \delta, \epsilon$}
  \KwResult{$x, f(x), i, err$}
  $v \leftarrow f(x_0);$\\
  \If{$ |v| < \epsilon $}{
    return (0, 0, 0, 1);
  }
  \Do{$ k < M $}{
    $ k\;+= 1 $ \\
    $ x_1 \leftarrow x_0 - \frac{v}{f'(x_0)}$ \\ 
    $ v \leftarrow f(x_1) $ \\
    \If{$ |x_1 - x_0| < \delta $ or $ |v| < \epsilon $}{
      return $ (x_1, v, i, 0) $;
    }
    $ x_0 \leftarrow x_1 $
  }
\end{algorithm}

\ \\
Metoda Newtona polega na linearyzacji wykresu w oparciu o wzór Taylora, tj.
$$ f(x) = f(c) + f'(c)(x - c) + \frac{1}{2!}f''(c)(x - c)^{2} + \dots $$
$$ l(x) = f(c) + f'(c)(x - c) $$
Jest możliwa do zastosowania na przedziale spełniającym następujące warunki:
\begin{enumerate}
  \item w przedziale tym powinien znajdować się dokładnie jeden pierwiastek
  \item wartości funkcji na jego krańcach muszą być różnych znaków
  \item pierwsza i druga pochodna funkcji w tym przedziale mają mieć stały znak
\end{enumerate} 
Na początku wybierany jest punkt $ x_0 $, następnie wyliczana jest styczna do funkcji w punkcie $ x_0 $. (do wyliczenia stycznej do funkcji potrzebna jest znajomość pochodnej funkcji) Wybierany jest kolejny punkt, $ x_1 $ - punkt przecięcia stycznej z osią $ OX $. Punkt $ x_1 $ staje się nowym bazowym, na którym wykonujywana jest operację analogiczną, jak do $ x_0 $. Wyznaczana jest kolejna styczna, jej przecięcie z osią odciętych staje się kolejnym punktem. Wyznaczając kolejne punkty, punkt przecięcia zbliża się do punktu zerowego. Operacja jest powtarzana aż do osiągnięcia wymaganej dokładności. 
\\
Metoda Newtona jest metodą o lokalnej zbieżności - aby móc wyznaczyć poprawne miejsce zerowe, należy dobrać odpowiednie parametry startowe - punkty początkowe $x_0$ oraz $x_1$. \\


\subsection{Implementacja}
Moja implementacja algorytmu w języku Julia znajduje się w module $ \texttt{MyModule} $ znajdującym się w pliku załączonym do tego sprawozdania.