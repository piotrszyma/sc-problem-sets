\section{Zadanie 3}
\subsection{Opis problemu}
Zadanie polegało na zaimplementowaniu algorytmu rozwiązującego równanie $ f(x) = 0 $ $\textsc{metodą siecznych} $.
\subsection{Analiza}
\begin{algorithm}[H]
  \KwData{$f, a, b, M, \delta, \epsilon$}
  \KwResult{$x, f(x), i, err$}
  $f_a \leftarrow f(a);$ $f_b \leftarrow f(b)$\\
  \Do{$ k < M $}{
    $ k \leftarrow k + 1 $ \\
    \If{$ |f_a| > |f_b| $}{
      $a \leftrightarrow b$; $f(a) \leftrightarrow f(b)$ 
    }
    $ s \leftarrow \frac{b - a}{f_b - f_a}$ \\
    $ b \leftarrow a$; $ f_b \leftarrow f_a$  \\ 
    $ a \leftarrow a - f_a * s $  \\ 

    \If{$ |b - a| < \delta \lor |f_a| < \epsilon $}{
      return $ (a, f_a, k, 0) $;
    }
  }
  return $ (a, f_a, M, 1) $;  
\end{algorithm}

\ \\
Metoda siecznych polega na wyznaczaniu miejsca zerowego w oparciu o wznaczanie punktów będących przecięciem pewnych siecznych funkcji. Algorytm zaczyna obliczenia od dwóch punktów początkowych $ x_0, x_1 $. Sieczna funkcji, przechodząca przez te punkty, przecina również oś OX. Z punktu przecięcia $ x_2 $ oraz punktu $ x_1 $ wyznaczana jest kolejna sieczna. Generowane sieczne przecinają oś odciętych w miejscach zbliżających się do szukanego miejsca zerowego funkcji. Operacja jest powtarzana aż do momentu osiągnięcia oczekiwanej dokładności. Metoda ta jest lokalnie zbieżna.
\subsection{Implementacja}
Moja implementacja algorytmu w języku Julia znajduje się w module $ \texttt{MyModule} $ znajdującym się w pliku załączonym do tego sprawozdania.