\section{Zadanie 4}
\subsection{Opis problemu}
Zadanie polegało na wyznaczeniu pierwiastka równania $ \sin{x} - (\frac{1}{2}x)^{2} = 0 $ przy pomocy wcześniej zaprogramowanych metod, z parametrami odpowiednio:
\begin{enumerate}
  \item dla metody bisekcji z przedziałem $[1.5, 2]$, $\delta = \frac{1}{2}*10^{-5}$, $\epsilon = \frac{1}{2}*10^{-5}$
  \item dla metody stycznych z $ x_0 = 1.5$, $\delta = \frac{1}{2}*10^{-5}$, $\epsilon = \frac{1}{2}*10^{-5}$  
  \item dla metody siecznych z $ x_0 = 1$, $x_1 = 2$, $\delta = \frac{1}{2}*10^{-5}$, $\epsilon = \frac{1}{2}*10^{-5}$
  
  \end{enumerate}
\subsection{Rozwiązanie}
Wykorzystując funkcje zaimplementowane w module $ \texttt{MyModule} $ wyliczyłem pierwiastki za pomocą  tych trzech algorytmów.
\subsection{Wynik}


Wyniki zestawiłem w tabeli poniżej: \\
\begin{center}
\begin{tabular}{|c|c|c|c|c|}
  \hline 
    Metoda & $x$ & $ f(x)$ & $i$ & $err$ \\
  \hline
  Bisekcja & 1.9337539672851562 & -2.7027680138402843e-7 & 16 & 0\\
  \hline 
  Stycznych & 1.933753779789742 & -2.2423316314856834e-8 & 4 & 0\\
  \hline  
  Siecznych & 1.933753644474301 & 1.564525129449379e-7 & 4 & 0 \\
  \hline
\end{tabular} 
\end{center}