\section{Zadanie 1}
\subsection{Opis problemu}
Zadanie polegało na zaimplementowaniu algorytmu rozwiązującego równanie $ f(x) = 0 $ $\textsc{metodą bisekcji} $.
\subsection{Pseudokod}
\begin{algorithm}[H]
  \KwData{$f, a, b, \delta, \epsilon$}
  \KwResult{$x, f(x), i, err$}
  $u \leftarrow f(a);$ $ v \leftarrow f(b) $ \\
  $e \leftarrow b - a $ \\
  \If{$ sgn(u) = sgn(v) $}{
    return (0, 0, 0, 1);
  }
  \While{true}{
    $ i\;= i + 1 $ \\
    $ e \leftarrow \frac{e}{2} $ \\ 
    $ c \leftarrow a + e $ \\
    $ w \leftarrow f(c) $ \\
    \If{$ |e| < \delta $ or $ |w| < \epsilon $}{
      return $ (w, f(x), i, 0) $;
    }
    \eIf{$ sgn(w) \neq sgn(u) $}{
      $ b \leftarrow c; $ $ v \leftarrow w $ \\
    }{
      $ a \leftarrow c; $ $ u \leftarrow w $ \\
    }
  }
\end{algorithm}
\subsection{Implementacja}
Moja implementacja algorytmu w języku Julia znajduje się w module $ \texttt{MyModule} $ znajdującym się w pliku załączonym do tego sprawozdania.