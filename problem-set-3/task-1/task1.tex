\section{Zadanie 1}
\subsection{Opis problemu}
Zadanie polegało na zaimplementowaniu algorytmu rozwiązującego równanie $ f(x) = 0 $ $\textsc{metodą bisekcji} $.
\subsection{Analiza}
\begin{algorithm}[H]
  \KwData{$f, a, b, \delta, \epsilon$}
  \KwResult{$x, f(x), i, err$}
  $u \leftarrow f(a);$ $ v \leftarrow f(b) $ \\
  $e \leftarrow b - a $ \\
  \If{$ sgn(u) = sgn(v) $}{
    return (0, 0, 0, 1);
  }
  \While{true}{
    $ i\;= i + 1 $ \\
    $ e \leftarrow \frac{e}{2} $ \\ 
    $ c \leftarrow a + e $ \\
    $ w \leftarrow f(c) $ \\
    \If{$ |e| < \delta $ or $ |w| < \epsilon $}{
      return $ (w, f(x), i, 0) $;
    }
    \eIf{$ sgn(w) \neq sgn(u) $}{
      $ b \leftarrow c; $ $ v \leftarrow w $ \\
    }{
      $ a \leftarrow c; $ $ u \leftarrow w $ \\
    }
  }
\end{algorithm}

\ \\
Jeśli $f$ jest funkcją ciągłą w przedziale $[a, b]$ i jeśli $f(a)f(b) < 0$, a więc $ f $ zmienia znak w $ [a, b] $ to ta funkcja musi mieć zero w $(a, b)$. Jest to konsekwencja własności Darboux funkcji ciągłych. Metoda bisekcji korzysta z tej własności. Jeśli $ f(a)f(b) < 0 $, to obliczamy $c = \frac{a + b}{2}$ i sprawdzamy, czy $f(a)f(c) < 0 $. Jesli tak, to $f$ ma zero w $[a, c]$; w przeciwnym wypadku jest $f(c)f(b) < 0$; wtedy pod $a$ podstawiamy $c$. W obu przypadkach przedział $[a, b]$, dwa razy krótszy od poprzedniego, zawiera zero funkcji $f$, więc postępowanie można powtórzyć.\\\\
Warto zwrócić uwagę na pewne istotne elementy algorytmu. Po pierwsze punkt środkowy $ c $ nie jest obliczany standardowo, tj. $ c = \frac{a + b}{2} $, ponieważ nową wielkość lepiej jest obliczać dodając do poprzedniej małą poprawkę, czyli $ c = a + \frac{b - a}{2} $ Kolejnym ważnym czynnikiem jest sprawdzanie znaku - tj. złym sposobem jest sprawdzanie go poprzez $ wu < 0 $, lepiej $sgn(u) \neq sgn(w) $. Nie można również zapomnieć o pewnej tolerancji błędu, zarówno od strony argumentów $\delta$, jak i wartości $\epsilon$. Warto również zastosować ograniczenie $ M $ maksymalnej ilości kroków. W tym wypadku jednak w treści zadania zostały nałożone inne ograniczenia.


\subsection{Implementacja}
Moja implementacja algorytmu w języku Julia znajduje się w module $ \texttt{MyModule} $ znajdującym się w pliku załączonym do tego sprawozdania. 