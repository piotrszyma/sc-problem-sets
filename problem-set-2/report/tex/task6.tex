\section{Zadanie 6}
\subsection{Opis problemu}
Dla zadanego równania rekurencyjnego przeprowadź serię eksperymentów.
$$ x_{n+1} = x_{n}^2 + c\; dla\; n = 0, 1, 2,\ldots$$
Wykonać w arytmetyce $ \mathtt{Float64} $ 40 iteracji wyrażenia dla 7 zestawów danych:
\begin{enumerate}
  \item $ c = -2 \;i\; x_0 = 1 $
  \item $ c = -2 \;i\; x_0 = 2 $
  \item $ c = -2 \;i\; x_0 = 1.99999999999999 $
  \item $ c = -1 \;i\; x_0 = 1 $
  \item $ c = -1 \;i\; x_0 = -1 $
  \item $ c = -1 \;i\; x_0 = 0.75 $
  \item $ c = -1 \;i\; x_0 = 0.25 $
\end{enumerate}
\subsection{Rozwiązanie}
W celu wyliczenia zadanych rekurencji stworzyłem program w Julii, którego kod źródłowy dołączyłem do sprawozdania. Wyniki obliczeń przedstawiłem w tabeli poniżej.
\subsection{Wynik}
Spośród wszystkich zestawów danych możemy wybrać te stabilne oraz te niestabilne. \\\\
W przypadku pierwszego zestawu danych, tj. $ x_0 = 1 $ i $ c = -2 $, wyniki obliczeń pokrywają się z przewidywaniami. Każda iteracja zwraca $ -1 $. Drugi zestaw również nie przynosi zaskoczenia - każdy wynik wynosi $ -2 $. W trzecim zestawie danych operujemy na liczbie o 14 cyfrach po przecinku - jej mnożenie generuje liczbę, która nie mieści się już w arytmetyce $ \mathtt{Float64} $ i dochodzi do przybliżenia - pojawia się błąd. Każde kolejne mnożenie powoduje kolejne przybliżenie - błąd narasta. Taki algorytm z numerycznego punktu widzenia zalicza się do grupy niestabilnych i jest niepożądany. \\\\
Ciągi $ 4 $ oraz $ 5 $ tworzą cykl składający się z liczb całkowitych, w ich przypadku nigdy nie dojdzie do wyskoczenia poza precyzję arytmetyki. Te zestawy danych są stabilne.\\\\
Ostatnie dwa ciągi są o tyle interesujące, że wartość, do której - z matematycznego punktu widzenia - dążą, osiągana jest już na wysokości $ ~13 $ iteracji. Wynika to z ograniczeń precyzji arytmetyki. Liczby są na tyle małe, że arytmetyka wymaga ich zaokrąglenia. \\\\
W celu określenia stabilności danego zestawu danych możemy posłużyć się iteracją graficzną. Przykładowe iteracje możemy znaleźć w książce „Granice chaosu” (fig 1.35, fig 1.36). Te zestawy, których graficzne iteracje będą skupiały się wokół jednego punktu, są stabilne. Zestawy, których iteracje „rozbijają się” na cały wykres, charakteryzują się niestabilnością. \\\\
Poniżej załączyłem tabele z wynikami.
\begin{center}
  \subsubsection*{Float64}
$$
\begin{array}{c|c|c|c|c}
algorytm & przed & \sigma & po & \sigma\\
\hline
1 & 1.0251881368e-10 & -1.1184955314e+01 & -4.2963427399e-03 & -4.2682954616e+08 \\
2 & -1.5643308870e-10 & -1.4541186645e+01 & -4.2963429987e-03 & -4.2682957187e+08 \\
3 & 0.0000000000e+00 & -1.0000000000e+00 & -4.2963428423e-03 & -4.2682955633e+08 \\
4 & 0.0000000000e+00 & -1.0000000000e+00 & -4.2963428423e-03 & -4.2682955633e+08 \\
\end{array}
$$
\subsubsection*{Float32}
$$
\begin{array}{c|c|c|c|c}
algorytm & przed & \sigma & po & \sigma\\
\hline
1 & -4.9994429946e-01 & -4.9668057661e+10 & -4.9994429946e-01 & -4.9668057661e+10 \\
2 & -4.5434570313e-01 & -4.5137965580e+10 & -4.5434570313e-01 & -4.5137965580e+10 \\
3 & -5.0000000000e-01 & -4.9673591353e+10 & -5.0000000000e-01 & -4.9673591353e+10 \\
4 & -5.0000000000e-01 & -4.9673591353e+10 & -5.0000000000e-01 & -4.9673591353e+10 \\
\end{array}
$$
  
\end{center}