\section{Zadanie 6}
\subsection{Opis problemu}
Dla zadanego równania rekurencyjnego przeprowadź serię eksperymentów.
$$ x_{n+1} = x_{n,2} + c\; dla\; n = 0, 1, 2,\ldots$$
Wykonać w arytmetyce $ \mathtt{Float64} $ 40 iteracji wyrażenia dla 7 zestawów danych:
\begin{enumerate}
  \item $ c = -2 \;i\; x_0 = 1 $
  \item $ c = -2 \;i\; x_0 = 2 $
  \item $ c = -2 \;i\; x_0 = 1.99999999999999 $
  \item $ c = -1 \;i\; x_0 = 1 $
  \item $ c = -1 \;i\; x_0 = -1 $
  \item $ c = -1 \;i\; x_0 = 0.75 $
  \item $ c = -1 \;i\; x_0 = 0.25 $
\end{enumerate}
\subsection{Rozwiązanie}
W celu wyliczenia zadanych rekurencji stworzyłem program w Julii, którego kod źródłowy dołączyłem do sprawozdania. Wyniki obliczeń przedstawiłem w tabeli poniżej.
\subsection{Wynik}
\begin{center}
  \subsubsection*{Float64}
$$
\begin{array}{c|c|c|c|c}
algorytm & przed & \sigma & po & \sigma\\
\hline
1 & 1.0251881368e-10 & -1.1184955314e+01 & -4.2963427399e-03 & -4.2682954616e+08 \\
2 & -1.5643308870e-10 & -1.4541186645e+01 & -4.2963429987e-03 & -4.2682957187e+08 \\
3 & 0.0000000000e+00 & -1.0000000000e+00 & -4.2963428423e-03 & -4.2682955633e+08 \\
4 & 0.0000000000e+00 & -1.0000000000e+00 & -4.2963428423e-03 & -4.2682955633e+08 \\
\end{array}
$$
\subsubsection*{Float32}
$$
\begin{array}{c|c|c|c|c}
algorytm & przed & \sigma & po & \sigma\\
\hline
1 & -4.9994429946e-01 & -4.9668057661e+10 & -4.9994429946e-01 & -4.9668057661e+10 \\
2 & -4.5434570313e-01 & -4.5137965580e+10 & -4.5434570313e-01 & -4.5137965580e+10 \\
3 & -5.0000000000e-01 & -4.9673591353e+10 & -5.0000000000e-01 & -4.9673591353e+10 \\
4 & -5.0000000000e-01 & -4.9673591353e+10 & -5.0000000000e-01 & -4.9673591353e+10 \\
\end{array}
$$
  
\end{center}

W przypadku pierwszego zestawu danych, tj. $ x_0 = 1 $ i $c = -2 $, wyniki obliczeń pokrywają się z przewidywaniami. Każda iteracja zwraca $ -1 $. Drugi zestaw również nie przynosi zaskoczenia - każdy wynik wynosi $ -2 $. W trzecim zestawie danych błędy zaczynają pojawiać się na wysokości 15 iteracji.

ANALIZA WYNIKU