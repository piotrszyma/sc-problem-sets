\section{Zadanie 5}
\subsection{Opis problemu}
Równanie rekurencyjne (model logistyczny, model wzrostu populacji)
$$ p_{n+1} := p_n + rp_n(1 - p_n)\; dla\; n = 0, 1, 2,\ldots$$

Zadanie polegało na przeprowadzeniu następującego eksperymentu:
\begin{enumerate}[label=(\alph*)]
  \item Dla danych $ p_0 = 0.01 $ oraz $ r = 3 $ wykonać 40 iteracji w/w wyrażenia (1), następnie 40 iteracji z modyfikacją - po 10 iteracji obciąć wynik (2), odrzucając cyfry po 3 miejscu po przecinku. Obliczenia wykonać w arytmetyce $ \mathtt{Float32} $. Porównać wyniki.
  \item Porównać wyliczanie 40 iteracji dla arytmetyk $ \mathtt{Float32} $ oraz $ \mathtt{Float64} $.
\end{enumerate}
\subsection{Rozwiązanie}
W celu wyliczenia zadanych rekurencji stworzyłem program w Julii. Kody źródłowe dołączyłem w osobnym pliku.
\subsection{Wynik}
Wyniki obliczeń przedstawiłem w tabeli poniżej. \\\\
W punkcie (a) do 10 iteracji wyniki nie różniły się od siebie. Po obcięciu cyfr znaczączych w przykładzie (2), kolejne iteracje 

\begin{center}
  \subsubsection*{Float64}
$$
\begin{array}{c|c|c|c|c}
algorytm & przed & \sigma & po & \sigma\\
\hline
1 & 1.0251881368e-10 & -1.1184955314e+01 & -4.2963427399e-03 & -4.2682954616e+08 \\
2 & -1.5643308870e-10 & -1.4541186645e+01 & -4.2963429987e-03 & -4.2682957187e+08 \\
3 & 0.0000000000e+00 & -1.0000000000e+00 & -4.2963428423e-03 & -4.2682955633e+08 \\
4 & 0.0000000000e+00 & -1.0000000000e+00 & -4.2963428423e-03 & -4.2682955633e+08 \\
\end{array}
$$
\subsubsection*{Float32}
$$
\begin{array}{c|c|c|c|c}
algorytm & przed & \sigma & po & \sigma\\
\hline
1 & -4.9994429946e-01 & -4.9668057661e+10 & -4.9994429946e-01 & -4.9668057661e+10 \\
2 & -4.5434570313e-01 & -4.5137965580e+10 & -4.5434570313e-01 & -4.5137965580e+10 \\
3 & -5.0000000000e-01 & -4.9673591353e+10 & -5.0000000000e-01 & -4.9673591353e+10 \\
4 & -5.0000000000e-01 & -4.9673591353e+10 & -5.0000000000e-01 & -4.9673591353e+10 \\
\end{array}
$$
  
\end{center}

ANALIZA WYNIKU