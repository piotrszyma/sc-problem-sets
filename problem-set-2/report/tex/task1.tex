\section{Zadanie 1}
\subsection{Opis problemu}
Powtórz zadanie 5 z listy 1, ale usuń ostatnią 9 z $ x_4 $ i ostatnią 7 z $ x_5 $. Jaki wpływ na
wyniki mają niewielkie zmiany danych?
\subsection{Rozwiązanie}
Uruchomiłem program z pierwszej listy po dokonaniu modyfikacji wejścia. Wyniki zestawiłem w tabeli poniżej. Numeracja algorytmów to odpowiednio 1, 2 - algorytmy sumowania odpowiednio w górę i w dół, 3, 4 - sumy częściowe.
\subsection{Wynik}
\begin{center}
\subsubsection*{Float64}
$$
\begin{array}{c|c|c|c|c}
algorytm & przed & \sigma & po & \sigma\\
\hline
1 & 1.0251881368e-10 & -1.1184955314e+01 & -4.2963427399e-03 & -4.2682954616e+08 \\
2 & -1.5643308870e-10 & -1.4541186645e+01 & -4.2963429987e-03 & -4.2682957187e+08 \\
3 & 0.0000000000e+00 & -1.0000000000e+00 & -4.2963428423e-03 & -4.2682955633e+08 \\
4 & 0.0000000000e+00 & -1.0000000000e+00 & -4.2963428423e-03 & -4.2682955633e+08 \\
\end{array}
$$
\subsubsection*{Float32}
$$
\begin{array}{c|c|c|c|c}
algorytm & przed & \sigma & po & \sigma\\
\hline
1 & -4.9994429946e-01 & -4.9668057661e+10 & -4.9994429946e-01 & -4.9668057661e+10 \\
2 & -4.5434570313e-01 & -4.5137965580e+10 & -4.5434570313e-01 & -4.5137965580e+10 \\
3 & -5.0000000000e-01 & -4.9673591353e+10 & -5.0000000000e-01 & -4.9673591353e+10 \\
4 & -5.0000000000e-01 & -4.9673591353e+10 & -5.0000000000e-01 & -4.9673591353e+10 \\
\end{array}
$$

\end{center}
W przypadku arytmetyki Float64 lekka zmiana wektorów przyczyniła się do zmiany ostatecznego wyniku - wciąż znacznie odbiegajacego od realnych wartości. Dla arytmetyki Float32 usunięcie ostatnich cyfr nie miało wpływu na ostateczny wynik, gdyż arytmetyka nie jest wystarczająco dokładna.

ANALIZA WYNIKU
