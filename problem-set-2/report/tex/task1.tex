\section{Zadanie 1}
\subsection{Opis problemu}
Powtórz zadanie 5 z listy 1, ale usuń ostatnią 9 z $ x_4 $ i ostatnią 7 z $ x_5 $. Jaki wpływ na
wyniki mają niewielkie zmiany danych?
\subsection{Rozwiązanie}
Uruchomiłem program z pierwszej listy po dokonaniu modyfikacji wejścia. Wyniki zestawiłem w tabeli poniżej. Numeracja algorytmów to odpowiednio 1, 2 - algorytmy sumowania odpowiednio w górę i w dół, 3, 4 - sumy częściowe.
\subsection{Wynik}
\begin{center}
Wyniki (a)
$$
\begin{array}{c|c|c|c}
k & |P(z_k)| & |p(z_k)| & |z_k - k|\\
\hline
1 & 3.6352000000e+04 & 3.8400000000e+04 & 3.0109248428e-13\\
2 & 1.8176000000e+05 & 1.9814400000e+05 & 2.8318236645e-11\\
3 & 2.0940800000e+05 & 3.0156800000e+05 & 4.0790348876e-10\\
4 & 3.1068160000e+06 & 2.8446720000e+06 & 1.6262468261e-08\\
5 & 2.4114688000e+07 & 2.3346688000e+07 & 6.6576979130e-07\\
6 & 1.2015206400e+08 & 1.1882496000e+08 & 1.0754175227e-05\\
7 & 4.8039833600e+08 & 4.7829094400e+08 & 1.0200279301e-04\\
8 & 1.6826910720e+09 & 1.6784972800e+09 & 6.4417039224e-04\\
9 & 4.4653265920e+09 & 4.4578595840e+09 & 2.9152943621e-03\\
10 & 1.2707126784e+10 & 1.2696907264e+10 & 9.5869575183e-03\\
11 & 3.5759895552e+10 & 3.5743469056e+10 & 2.5022932909e-02\\
12 & 7.2167715840e+10 & 7.2146650624e+10 & 4.6716746153e-02\\
13 & 2.1572362906e+11 & 2.1569633075e+11 & 7.4314032447e-02\\
14 & 3.6538325094e+11 & 3.6534479360e+11 & 8.5244408198e-02\\
15 & 6.1398775347e+11 & 6.1393841562e+11 & 7.5493799699e-02\\
16 & 1.5550277519e+12 & 1.5549610972e+12 & 5.3713283392e-02\\
17 & 3.7776237783e+12 & 3.7775329469e+12 & 2.5427146237e-02\\
18 & 7.1995548611e+12 & 7.1994474752e+12 & 9.0786472835e-03\\
19 & 1.0278376163e+13 & 1.0278235657e+13 & 1.9098182994e-03\\
20 & 2.7462952745e+13 & 2.7462788907e+13 & 1.9070876336e-04\\
\end{array}
$$
Roots(P) (a)
$$
\begin{array}{c|c}
n & pierwiastek\\
\hline
1 & 0.9999999999996989\\
2 & 2.0000000000283182\\
3 & 2.9999999995920965\\
4 & 3.9999999837375317\\
5 & 5.000000665769791\\
6 & 5.999989245824773\\
7 & 7.000102002793008\\
8 & 7.999355829607762\\
9 & 9.002915294362053\\
10 & 9.990413042481725\\
11 & 11.025022932909318\\
12 & 11.953283253846857\\
13 & 13.07431403244734\\
14 & 13.914755591802127\\
15 & 15.075493799699476\\
16 & 15.946286716607972\\
17 & 17.025427146237412\\
18 & 17.99092135271648\\
19 & 19.00190981829944\\
20 & 19.999809291236637\\
\end{array}
$$
Wyniki (b)
$$
\begin{array}{c|c|c}
k & |P(z_k)| & |z_k - k|\\
\hline
1 & 2.0992000000e+04 & 1.6431300764e-13\\
2 & 3.4918400000e+05 & 5.5037308044e-11\\
3 & 2.2215680000e+06 & 3.3965799062e-09\\
4 & 1.0467840000e+07 & 8.9724362162e-08\\
5 & 3.9463936000e+07 & 1.4261120898e-06\\
6 & 1.2914841600e+08 & 2.0476673031e-05\\
7 & 3.8812313600e+08 & 3.9792957758e-04\\
8 & 1.0725473280e+09 & 7.7720290994e-03\\
9 & 3.0655754240e+09 & 8.4183632067e-02\\
10 & 7.1431136380e+09 & 6.5195868304e-01\\
11 & 7.1431136380e+09 & 1.1109180273e+00\\
12 & 3.3577561132e+10 & 1.6652812906e+00\\
13 & 3.3577561132e+10 & 2.0458202767e+00\\
14 & 1.0612064533e+11 & 2.5188358712e+00\\
15 & 1.0612064533e+11 & 2.7128805313e+00\\
16 & 3.3151034760e+11 & 2.9060018735e+00\\
17 & 3.3151034760e+11 & 2.8254835213e+00\\
18 & 9.5394246098e+12 & 2.4540214463e+00\\
19 & 9.5394246098e+12 & 2.0043294443e+00\\
20 & 1.1144535045e+13 & 8.4691021519e-01\\
\end{array}
$$
Roots(P) (b)
$$
\begin{array}{c|c}
n & pierwiastek\\
\hline
1 & 0.9999999999998357 + 0.0im\\
2 & 2.0000000000550373 + 0.0im\\
3 & 2.99999999660342 + 0.0im\\
4 & 4.000000089724362 + 0.0im\\
5 & 4.99999857388791 + 0.0im\\
6 & 6.000020476673031 + 0.0im\\
7 & 6.99960207042242 + 0.0im\\
8 & 8.007772029099446 + 0.0im\\
9 & 8.915816367932559 + 0.0im\\
10 & 10.095455630535774 - 0.6449328236240688im\\
11 & 10.095455630535774 + 0.6449328236240688im\\
12 & 11.793890586174369 - 1.6524771364075785im\\
13 & 11.793890586174369 + 1.6524771364075785im\\
14 & 13.992406684487216 - 2.5188244257108443im\\
15 & 13.992406684487216 + 2.5188244257108443im\\
16 & 16.73074487979267 - 2.812624896721978im\\
17 & 16.73074487979267 + 2.812624896721978im\\
18 & 19.5024423688181 - 1.940331978642903im\\
19 & 19.5024423688181 + 1.940331978642903im\\
20 & 20.84691021519479 + 0.0im\\
\end{array}
$$

\end{center}
W przypadku arytmetyki Float64 lekka zmiana wektorów przyczyniła się do zmiany ostatecznego wyniku - wciąż znacznie odbiegajacego od realnych wartości. Dla arytmetyki Float32 usunięcie ostatnich cyfr nie miało wpływu na ostateczny wynik, gdyż arytmetyka nie jest wystarczająco dokładna.

ANALIZA WYNIKU
