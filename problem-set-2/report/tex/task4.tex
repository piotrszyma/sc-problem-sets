\section{Zadanie 4}
\subsection{Opis problemu}
,,Złośliwy wielomian'' Wilkinsona. Zainstalować pakiet $ \mathtt{Polynomials} $.
\begin{enumerate}[label=(\alph*)]
  \item Użyć funkcji $ \mathtt{roots} $ z pakietu $ \mathtt{Polynomials} $ do obliczenia 20 zer wielomianu P zadanego w treści zadania. Sprawdzić obliczone pierwiastki $ z_k, 1 \leq k \leq 20 $, obliczając $|P(z_k)|$, $|p(z_k)|$ i $|z_k - k|$. Wyjaśnić rozbieżność. \\
  Zapoznać się z funkcjami $ \mathtt{Poly},\; \mathtt{poly},\; \mathtt{polyval}$ z pakietu $ \mathtt{Polynomials} $.
  \item Powtórzyć eksperyment Wilkinsona, tj. zmienić współczynnik $-210$ na $ -210-2^{-23}$. Wyjaśnić zjawisko.
\end{enumerate}
\subsection{Rozwiązanie}
Funkcja $ \mathtt{Poly(x)} $ generuje wielomian w zależności od współczynników podanych jako parametr.
Funkcja $ \mathtt{poly(x)} $ tworzy wielomian na podstawie pierwiastków wielomianu, natomiast funkcja $ \mathtt{polyval(p, x)} $ pozwala na wyliczenie wartości wielomianu $ \mathtt{p} $ dla zadanego argumentu $ \mathtt{x} $. Napisałem program z użyciem wyżej wymienionych funkcji z biblioteki $ \mathtt{Polynomials} $. Kody źródłowe dołaczyłem do sprawozdania. 

\subsection{Wynik}
Wyniki poszczególnych części zadań zamieściłem w tabelach poniżej - zatytułowanych odpowiednio (a) i (b). \\\\
Rozwiązując podpunkt (a) zadania zaobserwowałem pewne rozbieżności w wynikach. 
Porównując wartości $ P(z_k) $ oraz $ p(z_k) $ zauważyłem, że wyniki różnią się od siebie. 
Wyliczone za pomocą funkcji $ \mathtt{roots(P(x))} $ pierwiastki wielomianu $ P $ różnią się od jego rzeczywistych pierwiastków. Wnioskiem z poczynionych obserwacji jest fakt, że operacje wyznaczania wielomianu na podstawie współczynników czy tworzenia go z pierwiastków, są obarczone pewnym zaburzeniem, wynikającym z precyzji arytmetyki użytej do wykonania obliczeń. \\\\
Podpunkt (b) dotyczył złośliwego wielomianu Wilkinsona - zaburzania wielomianów tak, aby wykonywane na nich operacje numeryczne generowały duże rozbieżności od rzeczywistych wyników. Mała zmiana jednego ze współczynników sprawiła, że wyniki w podpunkcie (b) znacznie różniły się od punktu (a). Ta mała zmiana, przez którą wartość jednego z czynników wymagała dużej dokładności (aproksymowanej przez skończoną arytmetykę) wygenerowała duży błąd - przy każdej operacji korzystającej z tej liczby. Funkcja $ \mathtt{roots(P'(x))} $ wywołana na wielomianie z drugiego przykładu, zwróciła pierwiastki zespolone.
\newpage
\begin{center}
  \subsubsection*{Float64}
$$
\begin{array}{c|c|c|c|c}
algorytm & przed & \sigma & po & \sigma\\
\hline
1 & 1.0251881368e-10 & -1.1184955314e+01 & -4.2963427399e-03 & -4.2682954616e+08 \\
2 & -1.5643308870e-10 & -1.4541186645e+01 & -4.2963429987e-03 & -4.2682957187e+08 \\
3 & 0.0000000000e+00 & -1.0000000000e+00 & -4.2963428423e-03 & -4.2682955633e+08 \\
4 & 0.0000000000e+00 & -1.0000000000e+00 & -4.2963428423e-03 & -4.2682955633e+08 \\
\end{array}
$$
\subsubsection*{Float32}
$$
\begin{array}{c|c|c|c|c}
algorytm & przed & \sigma & po & \sigma\\
\hline
1 & -4.9994429946e-01 & -4.9668057661e+10 & -4.9994429946e-01 & -4.9668057661e+10 \\
2 & -4.5434570313e-01 & -4.5137965580e+10 & -4.5434570313e-01 & -4.5137965580e+10 \\
3 & -5.0000000000e-01 & -4.9673591353e+10 & -5.0000000000e-01 & -4.9673591353e+10 \\
4 & -5.0000000000e-01 & -4.9673591353e+10 & -5.0000000000e-01 & -4.9673591353e+10 \\
\end{array}
$$

\end{center}