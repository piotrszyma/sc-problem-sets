\section{Zadanie 3}
\subsection{Opis problemu}
Rozwiązać układy równań podane w specyfikacji zadania za pomocą dwóch algorytmów: eliminacji Gaussa ($ \mathtt{x=A/b} $) oraz $ x = A^{-1}b\; (\mathtt{x=inv(A)*b}) $. Eksperymenty wykonać dla macierzy Hilberta $\mathbf{H_n}$ z rosnącym stopniem n > 1 oraz dla macierzy losowej $R_n$, n = 5, 10, 20 z rosnącym wskaźnikiem uwarunkowania $ c = 10, 10^3, 10^7, 10^{12}, 10^{16} $. Porównać obliczony $\tilde{x}$ z rozwiązaniem dokładnym $ x = (1, . . . , 1)T $, tj. policzyć błędy względne. \\ \\
W języku Julia za pomocą funkcji $ \mathtt{cond(A)} $ można sprawdzić jaki jest wskaźnik uwarunkowania wygenerowanej macierzy. Natomiast za pomocą funkcji $ \mathtt{rank(A)} $ można sprawdzić jaki jest rząd macierzy.
\subsection{Rozwiązanie}
W celu wykonania podanych w treści zadania obliczeń, stworzyłem w języku Julia odpowiednie funkcje. Kody źródłowe programów dołączyłem do sprawozdania. Błędy względne obliczeń wyliczyłem przy pomocy normy wektora wg. wzoru:
$$ \delta = \frac{|| \widetilde{x} - x ||}{||x||} $$
\subsection{Wynik}
W przypadku macierzy losowych zaobserwowałem, że błąd bezwzględny, niezależnie od stopnia macierzy, oscyluje w granicach $ 10^{-16} $. Zupełnie inaczej wyglądało to w przypadku macierzy Hilberta. \\\\
Macierz Hilberta jest postaci $ h_{ij} = \frac{1}{i + j - 1}$. Jak widać, wiele z komórek macierzy nie da się zapisać w postaci skończonej w reprezentacji biarnej, np. $ \frac{1}{3}, \frac{1}{6} $. Macierz Hilberta jest macierzą źle uwarunkowaną - wykonując obliczenia z jej użyciem jesteśmy narażeni na duży wpływ błędu wynikajacego z reprezentacji numerycznej, dlatego rozwiązywanie w numeryczny sposób nawet małych układów równań z wykorzystaniem macierzy Hilberta jest niemożliwe. \\\\
Zaobserwowałem, że wraz ze wzorestem stopnia macierzy, zarówno obliczenia wykonywane metodą eliminacji Gaussa, jak i przy użyciu macierzy odwrotnej, rosną wraz ze wzrostem stopnia n macierzy. Błąd w przypadku metody eliminacji Gaussa jest jednak opatrzony mniejszym błędem, ponieważ algorytm ten (operujący na macierzy w postaci schodkowej) pozwala na dojscie do wyniku przy wykonaniu mniejszej ilości operacji. \\
W przypadku macierzy losowej o zadanym współczynniku metoda eliminacji Gaussa również okazała się lepsza. Tutaj również górę wzięła mniejsza ilość operacji. W tym wypadku zarówno wraz ze wzrostem rozmiarów macierzy, jak i wzorstem czynnika $ cond $ błąd rosnął. \\\\
Poniżej przedstawiłem tabelę - zestawienie wyników w postaci wyliczonych błędów względnych. \\
\begin{center}
\subsubsection*{Float64}
$$
\begin{array}{c|c|c|c|c}
algorytm & przed & \sigma & po & \sigma\\
\hline
1 & 1.0251881368e-10 & -1.1184955314e+01 & -4.2963427399e-03 & -4.2682954616e+08 \\
2 & -1.5643308870e-10 & -1.4541186645e+01 & -4.2963429987e-03 & -4.2682957187e+08 \\
3 & 0.0000000000e+00 & -1.0000000000e+00 & -4.2963428423e-03 & -4.2682955633e+08 \\
4 & 0.0000000000e+00 & -1.0000000000e+00 & -4.2963428423e-03 & -4.2682955633e+08 \\
\end{array}
$$
\subsubsection*{Float32}
$$
\begin{array}{c|c|c|c|c}
algorytm & przed & \sigma & po & \sigma\\
\hline
1 & -4.9994429946e-01 & -4.9668057661e+10 & -4.9994429946e-01 & -4.9668057661e+10 \\
2 & -4.5434570313e-01 & -4.5137965580e+10 & -4.5434570313e-01 & -4.5137965580e+10 \\
3 & -5.0000000000e-01 & -4.9673591353e+10 & -5.0000000000e-01 & -4.9673591353e+10 \\
4 & -5.0000000000e-01 & -4.9673591353e+10 & -5.0000000000e-01 & -4.9673591353e+10 \\
\end{array}
$$

\end{center}

