\section{Zadanie 2}
\subsection{Opis problemu}
Celem tego zadania było, dla macierzy o strukturze takiej, jak w zadaniu pierwszym, stworzyć metodę wyznaczającą rozkład LU przy pomocy metody eliminacji Gaussa. W tym wypadku również nalezało stworzyć dwa warianty:
\begin{enumerate}
  \item bez wyboru elementu głównego
  \item z wyborem elementu głównego
\end{enumerate}
\subsection{Analiza}
Rozkład LU to dwie macierze, L - będąca lewą dolną macierzą trójkątną, na indeksach $a_{kk}$ mającą wartości $1$, a U - prawą górną. Iloczyn tych macierzy winien dać wyjściową macierz, której rozkładu LU szukamy, tj.:
$$ M = LU $$
\subsection{Rozwiązanie}
Do znalezienia układu LU zadanej macierzy można posłużyć się metodą eliminacji Gaussa podczas pierwszej pętli zerującej lewą dolną macierz trójkątną. Macierz po wyzerowaniu staje się szukaną prawą górną macierzą U, natomiast ze współczynników $\alpha{ij}$, które używaliśmy do wyzerowania zawartości komórek, tworzymy kolejną macierz - używając macierzy jednostkowej - następnie wypełniając ją odpowiednimi mnożnikami. Tak utworzona macierz jest szukaną macierzą L.
\begin{center}
  \begin{nmatrix}
    U
  \end{nmatrix}
  =
  \begin{bmatrix}
      a_{1,1} & a_{1,2} & a_{1,3} & \dots & a_{1,n} \\
      0 & a_{2,2} & a_{2,3} & \dots & a_{2,n} \\
      0 & 0 & a_{3,3} & \dots & a_{2,n} \\
      \vdots & \vdots & \ddots & \vdots & \vdots \\
      0 & 0 & 0 & \dots & a_{n-1,n} \\
      0 & 0 & 0 & \dots & a_{n,n} \\
    \end{bmatrix}
  \end{center}

  \begin{center}
    \begin{nmatrix}
      L
    \end{nmatrix}
    =
    \begin{bmatrix}
        1 & 0 & 0 & \dots & 0 \\
        \alpha_{2,1} & 1 & 0 & \dots & 0 \\
        \alpha_{3,1} & \alpha_{3,2} & 1 & \dots & 0 \\
        \vdots & \vdots & \ddots & \vdots & \vdots \\
        \alpha_{n-1,1} & \alpha_{n-1,2} & \alpha_{n-1,3} & \dots & 1 \\
      \end{bmatrix}
    \end{center}