\section{Zadanie 1}
\subsection{Opis problemu}
Celem tego zadania było zaimplementowanie metody rozwiązującej układ $Ax = b$ metodą eliminacji Gaussa uwzględniając specyficzną postać macierzy $ A \in \mathbb{R}^{n \times n} $, tj. macierzy o charakterystycznej strukturze przedstawionej poniżej oraz wektora prawych stron $ b \in \mathbb{R}^{n} $, $n > 4$. \\
\[
\begin{bmatrix}
    A_1 & C_1 &  0  &  0  &  0  & \dots & 0 \\
    B_2 & A_2 & C_2 &  0  &  0  & \dots & 0 \\
     0  & B_3 & A_3 & C_3 &  0  & \dots & 0 \\
    \vdots  & \ddots & \ddots & \ddots & \ddots & \ddots & \vdots \\
    0  & \dots & 0 & B_{v-2} &  A_{v-2}  & C_{v-2} & 0 \\
    0  & \dots & 0 & 0 &  B_{v-1} & A_{v-1}  & C_{v-1}  \\
    0  & \dots & 0 & 0 &  0  & B_{v} &  A_{v} \\
\end{bmatrix}
\]
gdzie $v = \frac{n}{l}$, przy załozeniu, że $n$ jest podzielne przez $l$, a $l$ jest rozmiarem wszystkich bloków wewnętrznych. Bloki $ A_k \in \mathbb{R}^{l \times l} $ to macierze gęste, bloki $0$ to macierze zerowe stopnia $l$ bloki $ B_k \in \mathbb{R}^{l \times l} $ następującej postaci:
\begin{center}
\[
$ B_k $ = 
\begin{bmatrix}
    0 & \dots & 0 & b_{1 l-1}^{k} & b_{1 l}^{k} \\
    0 & \dots & 0 & b_{2 l-1}^{k} & b_{2 l}^{k} \\
    \vdots &  & \vdots & \vdots & \vdots \\
    0 & \dots & 0 & b_{l l-1}^{k} & b_{l l}^{k} \\
\end{bmatrix}
\]
\end{center}
Natomiast bloki $ C_k \in \mathbb{R}^{l \times l} $ to macierze diagonalne:
\begin{center}
\[
$ C_k $ = 
\begin{bmatrix}
    c_{1}^{k} & 0 & 0 & \dots & 0 \\
    0 & c_{2}^{k} & 0 & \dots & 0 \\
    \vdots & \ddots & \ddots & \ddots & \vdots \\
    0 & \dots & 0 & c_{l-1}^{k} & 0 \\
    0 & \dots & 0 & 0 & c_{l}^{k} \\

  \end{bmatrix}
\]
\end{center}
Należało stworzyć dwa warianty implementacji:
\begin{enumerate}
  \item bez wyboru elementu głównego
  \item z wyborem elementu głównego
\end{enumerate}
Dodatkowym wymaganiem była złożoność algorytmu, która - po uwzględnieniu postaci macierzy - ma wynosić nie, tak jak w standardowym algortytmie eliminacji $O(n^3)$, a $O(n)$.
\subsection{Analiza}
Istotą metody eliminacji Gaussa są dwa etapy - pierwszy z nich to sprowadzenie macierzy do postaci schodkowej, tj. takiej, w której niezerowe komórki znajdują się jedynie powyżej komórek o indeksie wiersza i kolumny równym sobie, następny rozwiązanie odpowiadającego tej macierzy układu równań.
\subsection{Rozwiązanie}
\subsubsection{Bez wyboru elementu głównego}
Pierwsza część algorytmu to manipulacja macierzą - za pomocą operacji elementarych - w celu doprowadzenia macierzy do postaci \textsc{trójkątnej górnej}. Idea tej części opiera się na: wyborze wiersza głównego, następnie mnożeniu go przez odpowiedni czynnik i dodawaniu do każdych kolejnych wierszy tak, by wiersze te w danej kolumnie zostały wyzerowane. 
$$ W_n = W_n + \alpha W_1 $$ 
\begin{center}
\begin{bmatrix}
    a_{1,1} & a_{1,2} & a_{1,3} & \dots & a_{1,n} \\
    0 & a_{2,2} & a_{2,3} & \dots & a_{2,n} \\
    \vdots & \vdots & \vdots & \vdots & \vdots \\
    0 & a_{n-1,2} & a_{2,3} & \dots & a_{n-1,n} \\
    0 & a_{n,2} & a_{n,3} & \dots & a_{n,n} \\
  \end{bmatrix}
\end{center}
W tym wypadku za wiersz główny został obrany wiersz $W_1$, a przy każdym innym wierszu współczynnik $ \alpha_n $ wynosił $ 0 = a_{i,1} - \alpha \;a_{1,1}$, tj. $ \alpha = \frac{a_{i,1}}{a_{1,1}} $
Po wykonaniu serii działań na kolejnych postaciach macierzy
$ A_1 \rightarrow A_2 \rightarrow \dots \rightarrow A_n $ otrzymujemy macierz trójkątną. 

\begin{center}
    \begin{bmatrix}
        a_{1,1} & a_{1,2} & a_{1,3} & \dots & a_{1,n} \\
        0 & a_{2,2} & a_{2,3} & \dots & a_{2,n} \\
        0 & 0 & a_{3,3} & \dots & a_{2,n} \\
        \vdots & \vdots & \ddots & \vdots & \vdots \\
        0 & 0 & 0 & \dots & a_{n-1,n} \\
        0 & 0 & 0 & \dots & a_{n,n} \\
      \end{bmatrix}
    \end{center}

Z tej macierzy (zaczynając od ostatniego wiesza ostatnej kolumny) jesteśmy wstanie wyznaczyć szukany wektor $X$ rozwiązań równania.
\subsubsection{Wybór elementu głównego}
W powyższym przykładzie musimy jednak poczynić pewne dodatkowe założenie - komórki $a_{kk}$, które używamy jako dzielniki w metodzie Gaussa nie mogą być równe zero. W przeciwnym wypadku dochodziłoby do dzielenia przez zero. Mało tego, ich wartości nie mogą być bliskie zero - dlatego doprowadzenie macierzy do postaci, w której komórki $a_{kk}$ mają jak największą wartość, pozwala poprawić numeryczną dokładność. Szukam więc wiersza - spośród wierszy $W_i$ t. że $i \in (k, \ldots, n)$. Wybór spośród wierszy $i < k$ zaburzyłby strukturę, do której dążymy - tj. zerową lewą dolną macierz trójkątną.
Będąc w tej sytuacji:
\begin{center}
    \begin{bmatrix}
        a_{1,1} & a_{1,2} & a_{1,3} & \dots & a_{1,n} \\
        0 & a_{2,2} & a_{2,3} & \dots & a_{2,n} \\
        0 & a_{3,2} & a_{3,3} & \dots & a_{3,n} \\
        \vdots & \vdots & \vdots & \vdots & \vdots \\
        0 & a_{n-1,2} & a_{2,3} & \dots & a_{n-1,n} \\
        0 & a_{n,2} & a_{n,3} & \dots & a_{n,n} \\
    \end{bmatrix}
\end{center}
Wybieramy komórkę o największej wartości spośród $a_{i, 2}$, $i \in (2, n) $, następnie zamieniamy miejscami wiersze $ W_2 $ i $ W_i $.
\subsection{Złożoność}
W rozpatrywanej w zadaniu macierzy, w przypadku eliminacji Gaussa, nie osiągniemy oczekiwanej liniowej złożności przy klasycznym sposobie eliminacji. Aby osiągnąć odpowiednią złożoność, do eliminacji należy wziąć jedynie niezerowe komórki, które znajdują się w blokach $ A $, $ B $ oraz $ C $, a więc zawężając - w każdej iteracji - zakres wierszy do wyzerowania kolumny do $ l $ - taka jest maksymalna rozpiętość niezerowych wartości, a zakres kolumn w danym wierszu do $ 3l $. Tym sposobem otrzymujemy złożoność $ O(n * l * 3l) $, a uwzględniając notację $O$ duże oraz fakt, że $l$ traktujemy jako stałą, dostajemy złożoność liniową.
W celu osiągniecia odpowiedniej złożoności pamięciowej, do przechowywania macierzy zostały użyte macierze typu $\texttt{SparseMatrixCSC}$, zaimplementowane z myślą o macierzach rzadkich, pamiętające tylko elementy niezerowe.