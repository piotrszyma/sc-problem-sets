\section{Zadanie 1}
\subsection{Opis problemu}
Celem tego zadania było zaimplementowanie metody rozwiązującej układ $Ax = b$ metodą eliminacji Gaussa uwzględniając specyficzną postać macierzy $ A \in \mathbb{R}^{n \times n} $, tj. macierzy o charakterystycznej strukturze przedstawionej poniżej oraz wektora prawych stron $ b \in \mathbb{R}^{n} $, $n > 4$. \\
\[
\begin{bmatrix}
    A_1 & C_1 &  0  &  0  &  0  & \dots & 0 \\
    B_2 & A_2 & C_2 &  0  &  0  & \dots & 0 \\
     0  & B_3 & A_3 & C_3 &  0  & \dots & 0 \\
    \vdots  & \ddots & \ddots & \ddots & \ddots & \ddots & \vdots \\
    0  & \dots & 0 & B_{v-2} &  A_{v-2}  & C_{v-2} & 0 \\
    0  & \dots & 0 & 0 &  B_{v-1} & A_{v-1}  & C_{v-1}  \\
    0  & \dots & 0 & 0 &  0  & B_{v} &  A_{v} \\
\end{bmatrix}
\]
gdzie $v = \frac{n}{l}$, przy załozeniu, że $n$ jest podzielne przez $l$, a $l$ jest rozmiarem wszystkich bloków wewnętrznych. Bloki $ A_k \in \mathbb{R}^{l \times l} $ to macierze gęste, bloki $0$ to macierze zerowe stopnia $l$ bloki $ B_k \in \mathbb{R}^{l \times l} $ następującej postaci:
\begin{center}
\[
$ B_k $ = 
\begin{bmatrix}
    0 & \dots & 0 & b_{1 l-1}^{k} & b_{1 l}^{k} \\
    0 & \dots & 0 & b_{2 l-1}^{k} & b_{2 l}^{k} \\
    \vdots &  & \vdots & \vdots & \vdots \\
    0 & \dots & 0 & b_{l l-1}^{k} & b_{l l}^{k} \\
\end{bmatrix}
\]
\end{center}
Natomiast bloki $ C_k \in \mathbb{R}^{l \times l} $ to macierze diagonalne:
\begin{center}
\[
$ C_k $ = 
\begin{bmatrix}
    c_{1}^{k} & 0 & 0 & \dots & 0 \\
    0 & c_{2}^{k} & 0 & \dots & 0 \\
    \vdots & \ddots & \ddots & \ddots & \vdots \\
    0 & \dots & 0 & c_{l-1}^{k} & 0 \\
    0 & \dots & 0 & 0 & c_{l}^{k} \\

  \end{bmatrix}
\]
\end{center}
Należało stworzyć dwa warianty implementacji:
\begin{enumerate}
  \item bez wyboru elementu głównego
  \item z wyborem elementu głównego
\end{enumerate}
Dodatkowym wymaganiem była złożoność algorytmu, która - po uwzględnieniu postaci macierzy - ma wynosić nie, tak jak w standardowym algortytmie eliminacji $O(n^3)$, a $O(n)$.
\subsection{Analiza}

\subsection{Implementacja}
